% Created 2023-12-15 vie 14:18
% Intended LaTeX compiler: pdflatex
\documentclass[11pt]{article}
\usepackage[utf8]{inputenc}
\usepackage{lmodern}
\usepackage[T1]{fontenc}
\usepackage[top=1in, bottom=1.in, left=1in, right=1in]{geometry}
\usepackage{graphicx}
\usepackage{longtable}
\usepackage{float}
\usepackage{wrapfig}
\usepackage{rotating}
\usepackage[normalem]{ulem}
\usepackage{amsmath}
\usepackage{textcomp}
\usepackage{marvosym}
\usepackage{wasysym}
\usepackage{amssymb}
\usepackage{amsmath}
\usepackage[theorems, skins]{tcolorbox}
\usepackage[version=3]{mhchem}
\usepackage[numbers,super,sort&compress]{natbib}
\usepackage{natmove}
\usepackage{url}
\usepackage[cache=false]{minted}
\usepackage[strings]{underscore}
\usepackage[linktocpage,pdfstartview=FitH,colorlinks,
linkcolor=blue,anchorcolor=blue,
citecolor=blue,filecolor=blue,menucolor=blue,urlcolor=blue]{hyperref}
\usepackage{attachfile}
\usepackage{setspace}
\usepackage{nacal}
\usepackage[spanish, ]{babel}
\usepackage{pdfpages}
\author{Marcos Bujosa}
\date{\today}
\title{Lección 1 del curso}
\begin{document}

\tableofcontents

\maketitle

\section{Vectores de \R[n] - T01}
\label{sec:org6b69d6d}


\subsection{Español}
\label{sec:orgafe8c69}


\subsubsection{Escena 1 - Definición de vector de \R[n]}
\label{sec:org29bd9e2}

\begin{minted}[frame=lines,fontsize=\scriptsize,linenos]{python}
<<Carga de la librería Manim y NacAL>>

<<copyright>>

class L01_01_VectoresDefinicion(VoiceoverScene):
    def construct(self):
        self.set_speech_service(GTTSService(lang="es", tld="com"))        
        myTemplate = TexTemplate()
        myTemplate.add_to_preamble(r"\usepackage{nacal}")

        # Copyright lateral
        <<copyrightLateral>>
        
        # Portada
        titulo = Title(r"Definición de vector de \R[n] y notación",
                     tex_template = myTemplate,
                     font_size=70).set_color(BLUE)
        self.play(Write(titulo))
        self.wait(1.5)
        self.play(FadeOut(titulo))
	
        # Definición de vector
        definicion = Tex("Un vector de ",
                         r"$\R[n]$",
                         r" es una \emph{lista ordenada} de $n$ números",
                         tex_template = myTemplate,
                         ).to_edge(UP).set_width(13)
        with self.voiceover(text=r"Un vector de R n es una lista ordenada de números.") as tracker:
            self.add(definicion)
	
        # Aclaración de la notación
        with self.voiceover(text=r"La R indica que los números son reales.") as tracker:
            self.play(Circumscribe(definicion[1][0], fade_out=True), run_time=tracker.duration)
            
        with self.voiceover(text=r"Y el superíndice n indica que la lista contiene n números.") as tracker:
            self.play(Circumscribe(definicion[1][1], fade_out=True), run_time=tracker.duration)
            self.wait(0.3)
            
        with self.voiceover(text=r"""que la lista sea ordenada
        significa que importa el orden en el que aparecen sus
        elementos.""") as tracker:
            self.play(Circumscribe(definicion[2][5:18], fade_out=True), run_time=tracker.duration)
	
        # Ejemplos
        Ej = Tex(r"\textbf{Ejemplos:}",
                 tex_template = myTemplate,
                 font_size=50).set_color(GREEN).next_to(definicion, DOWN, aligned_edge=LEFT)
        self.add(Ej)
	
        d = nc.Vector([1,2,3],'fila')
        Ej1 = MathTex( d.latex(), "\\ne", nc.Vector(reversed(d),'fila').latex() ) 
        with self.voiceover(text=r"Por ejemplo, los vectores 1 2 3 y 3 2 1 son distintos.") as tracker:
            self.play(FadeIn(Ej1))
            self.wait(tracker.duration-0.5)
            self.play(FadeOut(Ej1))
	
        p = nc.Vector([sp.pi,sp.pi,sp.pi,sp.pi],'fila')
        Ej2 = MathTex(p.latex(), r"\ne", nc.Vector(p|(1,2),'fila').latex() )
        with self.voiceover(text=r"""También son distintos los vectores con distinta cantidad de
        elementos. El vector de la izquierda pertenece a R 4, pues es
        una lista de 4 números reales, mientras que el de la derecha
        pertenece a R 2.""") as tracker:
            self.add(Ej2)
            self.wait(tracker.duration-0.5)
            self.play(FadeOut(Ej2))
	
        c = nc.Vector([sp.pi,nc.fracc(3,4),0,0.11],'fila')
        Ej3 = MathTex(c.latex(),"=",c.latex())
        with self.voiceover(text=r"""Por tanto dos vectores serán
        iguales, si y solo si, sus correspondientes listas son idénticas""") as tracker:
            self.play(FadeIn(Ej3))
            self.wait(tracker.duration)
            self.play(FadeOut(Ej3))
            
        self.play(FadeOut(Ej))
        self.wait(0.5)
        
\end{minted}

\subsubsection{Escena 2 - Notación vectores de \R[n]}
\label{sec:org7f7ce5d}

\begin{minted}[frame=lines,fontsize=\scriptsize,linenos]{python}
class L01_02_VectoresNotacion(VoiceoverScene):
    def construct(self):
        self.set_speech_service(GTTSService(lang="es", tld="com"))        
        myTemplate = TexTemplate()
        myTemplate.add_to_preamble(r"\usepackage{nacal}")
        
        # Copyright lateral
        <<copyrightLateral>>
        
        # Definición de vector
        definicion = Tex("Un vector de ",
                         r"$\R[n]$",
                         r" es una \emph{lista ordenada} de $n$ números",
                         tex_template = myTemplate,
                         ).to_edge(UP).set_width(13)
        self.add(definicion)
        d = nc.Vector([1,2,3],'fila')
        p = nc.Vector([sp.pi,sp.pi,sp.pi,sp.pi],'fila')
            
        # Notacion
        Notac = Tex(r"\textbf{Notación:}",
                 tex_template = myTemplate,
                 font_size=50).set_color(BLUE).next_to(definicion, DOWN, aligned_edge=LEFT)
        self.add(Notac)
        self.wait()

        a    = nc.Vector( [5, 1, 10] , 'fila')        
        Not1 = MathTex(a.latex(),        tex_template = myTemplate,)
        Not2 = MathTex(r"=",             tex_template = myTemplate,)
        Not3 = MathTex(a.copy().latex(), tex_template = myTemplate,)
        grp1 = VGroup(Not1,Not2,Not3).arrange(RIGHT)
        with self.voiceover(text=r"""Para expresar un vector basta indicar la lista de elementos en su
        correspondiente orden.""") as tracker:
            self.add(grp1)
            self.play(Circumscribe(definicion[2][5:18]),run_time=tracker.duration)
            
        with self.voiceover(text=r"""Por este motivo podemos escribir un mismo vector tanto en
        horizontal como en vertical (pero tenga en cuenta que muchos
        manuales no siguen este convenio de notación).""") as tracker:
            self.wait(tracker.duration+0.3)
	
        with self.voiceover(text=r"""Además, escribiremos la lista de
        números encerrada entre paréntesis; y pondremos una coma
        detrás de cada elemento cuando escribamos el vector en
        horizontal.""") as tracker:
            self.play(Indicate(grp1[0][0][::len(grp1[0][0])-1]),
                      Indicate(grp1[2][0][0:2]),
                      Indicate(grp1[2][0][-2:]),
                      run_time=tracker.duration/2)
            self.play(Flash(grp1[0][0][2]),
                      Flash(grp1[0][0][4]),
                      Flash(grp1[0][0][7]),
                      run_time=tracker.duration/8)
            self.wait(tracker.duration/8)
            self.play(Circumscribe(grp1[0]))
            self.play(FadeOut(grp1))
	
        VectorNoNumero =  MathTex(r"(3)",r"\ne",(3*nc.V1(1)).latex(),r"\in\R[1]", tex_template = myTemplate,)
        with self.voiceover(text=r"""Así podremos distinguir entre un número entre paréntesis y un
        vector de R 1 (que es una lista con un solo número).""") as tracker:
            self.add(VectorNoNumero)
            self.play(Indicate(VectorNoNumero[0]),run_time=tracker.duration/3)
            self.play(Indicate(VectorNoNumero[2]),run_time=tracker.duration/3)
            self.play(Indicate(VectorNoNumero[3]),
                      Flash(definicion[2][-9]),
                      run_time=tracker.duration/3)
            self.wait(0.5)
            self.play(FadeOut(VectorNoNumero))
            
        Vectores = MathTex(r"\Vect{a}, \Vect{b}, \Vect{c},\ldots\Vect{x}, \Vect{y}, \Vect{z}",
                           tex_template = myTemplate,).move_to( UP )
        Vector1  = MathTex(r"\Vect{a}=",a.copy().latex(),   tex_template = myTemplate,)
        Vector2  = MathTex(r"\Vect{d}=",d.copy().latex(),   tex_template = myTemplate,)
        Vector3  = MathTex(r"\Vect{x}=",p.copy().latex(),   tex_template = myTemplate,)
        grp3 = VGroup(Vector1,Vector2,Vector3).arrange(RIGHT, buff=2).next_to(Vectores, DOWN)
        with self.voiceover(text=r"Para denotar vectores usaremos letras minúsculas en negrita cursiva.") as tracker:
            self.add(Vectores)
            self.add(grp3)
            self.wait(tracker.duration/2)
            self.play(Indicate(Vectores),run_time=tracker.duration/2)
            self.play(FadeOut(Vectores))
            self.play(Indicate(Vector1[0][0],scale_factor=2.),
                      Indicate(Vector2[0][0],scale_factor=2.),
                      Indicate(Vector3[0][0],scale_factor=2.),
                      run_time=1.5)
            self.play(FadeOut(grp3))
            
        Vnulo = MathTex(r"\Vect{0}", tex_template = myTemplate,)#.move_to( UP )
        with self.voiceover(text=r"Un cero en negrita denota un vector cuyas componentes son todas nulas.") as tracker:
            self.add(Vnulo)
            self.play(Indicate(Vnulo))
            self.wait(tracker.duration/2)
            self.play(FadeOut(Vnulo))
	
        Vnulo1 = MathTex(r"\Vect{0}=", nc.V0(1).latex(), ",",  tex_template = myTemplate,)
        Vnulo2 = MathTex(r"\Vect{0}=", nc.V0(2).latex(), ",",  tex_template = myTemplate,)
        Vnulo3 = MathTex(r"\Vect{0}=", nc.V0(3).latex(), ",",  tex_template = myTemplate,)
        Vnulo6 = MathTex(r"\Vect{0}=", nc.V0(6).latex(), ",",  tex_template = myTemplate,)
        VnuloN = MathTex(r"\Vect{0}\in\R[100]",             tex_template = myTemplate,)
        grp2   = VGroup(Vnulo1,Vnulo2,Vnulo3,Vnulo6,VnuloN).arrange(RIGHT, buff=0.7)
        with self.voiceover( text = r"""Fíjese que un cero en negrita
        no indica su número de componentes. Normalmente la cantidad de
        ceros se deduce del contexto.""" ) as tracker:
            self.add(grp2)
            self.wait(tracker.duration)
            self.play(FadeOut(grp2),FadeOut(Notac),FadeOut(definicion))
            self.wait(1.5)

\end{minted}

\subsubsection{Escena 3 - Selección de elementos de un vector de \R[n]}
\label{sec:org6346092}

\begin{minted}[frame=lines,fontsize=\scriptsize,linenos]{python}
class L01_03_VectoresElementos(VoiceoverScene):
    def construct(self):
        self.set_speech_service(GTTSService(lang="es", tld="com"))
        
        myTemplate = TexTemplate()
        myTemplate.add_to_preamble(r"\usepackage{nacal}")
        
        # Copyright lateral
        <<copyrightLateral>>
        
        # Notacion
        Notac = Tex(r"\textbf{Notación para los elementos:}",
                 tex_template = myTemplate,
                 font_size=50).set_color(BLUE).to_corner(UL)
        self.wait()
        self.add(Notac)
        self.wait()
        
        # Elementos de un vector
        v_generico  = nc.Vector(sp.symbols('a:5')[1:],'fila')
        cs = MathTex(r"\Vect{a}=",
                     v_generico.latex(),
                     tex_template = myTemplate,)
        
        with self.voiceover(text = r"""Lo habitual es denotar cada
        elemento de un vector con la letra de su nombre sin negrita.""" ) as tracker:
            self.wait()
            self.play(FadeIn(cs), run_time=0.5)
            self.play( Circumscribe(cs[1][1]),
                       Circumscribe(cs[1][4]),
                       Circumscribe(cs[1][7]),
                       Circumscribe(cs[1][10]),
                       run_time=tracker.duration/2)
            
        with self.voiceover(text = r"""indicando con un subíndice su posición en la lista.""" ) as tracker:
            self.play( Flash(cs[1][2]),
                       Flash(cs[1][5]),
                       Flash(cs[1][8]),
                       Flash(cs[1][11]),
                       run_time=tracker.duration)
            self.play(FadeOut(cs))
            
        c = nc.Vector([sp.pi,nc.fracc(3,4),0,0.11],'fila')
        vector_c = MathTex(r"\Vect{c}=",c.latex(),tex_template = myTemplate,)
        A = VGroup(*[ MathTex("c_"+str(i+1)+"=&"+sp.latex(e))  for i,e in enumerate(c.lista)
                     ]).arrange(DOWN,aligned_edge=LEFT, buff=.5)
        B = Brace(A, LEFT)
        C = VGroup(A,B)        
        Elementos_c   = VGroup(vector_c, C).arrange(RIGHT, buff=1)
        with self.voiceover(text = r"""Así, para el vector C """) as tracker:
            self.play(FadeIn(vector_c))
            self.play(GrowFromCenter(B),FadeIn(A))
            
        with self.voiceover(text = r"""con c 1 denotamos su primera componente""") as tracker:
            self.play( Indicate(vector_c[1][1]),    Indicate(A[0]) )
        with self.voiceover(text = r"""con c 2 la segunda""") as tracker:
            self.play( Indicate(vector_c[1][3:6]),  Indicate(A[1]) )
        with self.voiceover(text = r"""y del mismo modo""") as tracker:
            self.play( Indicate(vector_c[1][7]),    Indicate(A[2]) )
        with self.voiceover(text = r"""con el resto de componentes""") as tracker:
            self.play( Indicate(vector_c[1][9:13]), Indicate(A[3]) )
            self.wait(0.5)
            self.play( FadeOut(vector_c), FadeOut(B), FadeOut(A) )
            self.wait(0.5)

        with self.voiceover(text = r"""El hecho de emplear dos tipos
        de fuentes:""" ) as tracker:
            self.add(cs)
            self.wait(tracker.duration)
            
        with self.voiceover(text = r"""la negrita cursiva para el vector""" ) as tracker:
            self.play( Circumscribe(cs[0][0]))
            
        with self.voiceover(text = r"""y solo la cursiva para sus
        componentes, limita la operatividad de esta tradicional
        notación.""" ) as tracker:
            self.play( Circumscribe(cs[1][1]),
                       Circumscribe(cs[1][4]),
                       Circumscribe(cs[1][7]),
                       Circumscribe(cs[1][10]))

        MTa = MathTex(r"\eleVR{a}{i}",tex_template = myTemplate).scale(3)
        MTb = MathTex(r"{a}_{i}=",tex_template = myTemplate).scale(3).next_to(MTa, LEFT)
        VG  = VGroup(MTb,MTa) 
        with self.voiceover(text = r"""Es preferible una notación que
        indique la selección de elementos mediante un operador.""" ) as tracker:
            self.play(cs.animate.to_corner(DL), run_time=tracker.duration/2)
            self.play(Indicate(VG[1][0][1]))

        def VectorGenerico(s,n):
            elem = lambda s,i: sp.Symbol(r'\eleVR{'+ s +'}{'+ str(i) + '}')
            return nc.Vector([elem(s,i) for i in range(1,n+1)], 'fila')
        
        v_generico2 = VectorGenerico('a',4)
        cs2 = MathTex(r"=",
                     v_generico2.latex(),
                     tex_template = myTemplate,).next_to(cs, RIGHT)
        csG  = VGroup(cs,cs2) 
        
        VGB = VGroup(*[MathTex(sp.latex(e) + "=\; & \eleVR{a}{" + str(i+1) + "}",
                               tex_template = myTemplate)
                       for i,e in enumerate(v_generico.lista)
                       ]).scale(3)
        
        with self.voiceover( text = r"""Al escribir como subíndice una
        barra vertical que media entre el vector y un índice,
        indicamos la operación consistente en seleccionar una
        componente, la indicada con dicho índice""" ) as tracker:
            self.play(FadeIn(VG[1]))
            self.wait(0.3)
            self.play(Indicate(VG[1][0][1:]))
            self.wait(2*tracker.duration/3)
            self.play(Indicate(VG[1][0][-1]))
            self.play(Write(VG[0]))
            self.wait()
            self.play(VG.animate.move_to([0,0,0]))
            self.play(Transform(VG[1][0][-1],VGB[0][0][-1]),
                      Transform(VG[0][0][:2],VGB[0][0][:2]), run_time=1.5)
            self.play( FadeIn(cs2) )
            self.play(FadeTransform(VGB[0][0][0:2],cs[1][ 1: 3]),
                      FadeTransform(VGB[0][0][3:],cs2[1][ 1: 6]), FadeOut(VG), run_time=1.5)
            self.play(FadeIn(VGB[1]), FadeOut(VGB[0][0][2]))
            self.play(FadeTransform(VGB[1][0][0:2],cs[1] [4: 6]),
                      FadeTransform(VGB[1][0][3:],cs2[1][ 7:12]), FadeOut(VGB[1][0][2]), run_time=1.5)
            self.play(FadeIn(VGB[2]))
            self.play(FadeTransform(VGB[2][0][0:2],cs[1][ 7: 9]),
                      FadeTransform(VGB[2][0][3:],cs2[1][13:18]), FadeOut(VGB[2][0][2]), run_time=1.5)
            self.play(FadeIn(VGB[3]))
            self.play(FadeTransform(VGB[3][0][0:2],cs[1][10:12]),
                      FadeTransform(VGB[3][0][3:],cs2[1][19:24]), FadeOut(VGB[3][0][2]), run_time=1.5)
            
        with self.voiceover( text = r"""La notación tradicional distingue entre vectores y sus componentes
        usando fuentes con y sin negrita. Con frecuencia es dificil
        ver la diferencia a primera vista. Ese problema desaparece
        cuando usamos la notación con el operador selector""" ) as tracker:
            self.play(csG.animate.move_to([0,0,0]),  run_time=tracker.duration/4)
            self.play( Indicate(cs[1][1]),
                       Indicate(cs[1][4]),
                       Indicate(cs[1][7]),
                       Indicate(cs[1][10]),
                       Indicate(csG[0][0][0]), run_time=tracker.duration/2)
            self.play( Circumscribe(csG[1][1:]),   run_time=tracker.duration/4)
            self.play(FadeOut(Notac),FadeOut(cs),FadeOut(cs2))

        MTLR = MathTex(r"\eleVR{x}{i}",r"\;=\eleVL{x}{i}",tex_template = myTemplate).scale(3)
        with self.voiceover( text = r"""Además aceptaremos la selección de componentes operando tanto por la
        derecha como por la izquierda.""" ) as tracker:   
            self.play(FadeIn(MTLR[0]), run_time=2*tracker.duration/3)
            self.play(FadeIn(MTLR[1]))
            self.wait(tracker.duration/3+0.5)
            self.play(FadeOut(MTLR))
            
\end{minted}


\subsection{Versión en inglés}
\label{sec:orgeb1ea68}

\begin{minted}[frame=lines,fontsize=\scriptsize,linenos]{sh}
manim_render_translation $fichero.py -s $escena -d $escenaENG -l en -ql
\end{minted}


\section{Trozos comunes de código}
\label{sec:orgc0c1570}


\subsection{Carga de la librería Manim y NacAL}
\label{sec:orgd5df353}

\begin{minted}[frame=lines,fontsize=\scriptsize,linenos]{python}
from manim import *
from manim_voiceover import VoiceoverScene
from manim_voiceover.services.gtts import GTTSService
import nacal as nc
import sympy as sp

# PARA LA TRADUCCIÓN (pero no me ha funcionado)

#from manim_voiceover.translate import get_gettext
# # It is good practice to get the LOCALE and DOMAIN from environment variables
#import os
#LOCALE = os.getenv("LOCALE")
#DOMAIN = os.getenv("DOMAIN")
# The following function uses LOCALE and DOMAIN to set the language, and
# returns a gettext function that is used to insert translations.
#_ = get_gettext()
\end{minted}


\subsection{Creditos}
\label{sec:orgbccfcd4}

\begin{minted}[frame=lines,fontsize=\scriptsize,linenos]{python}
copyright = Tex(r"Copyright \textcopyright{\;} Marcos Bujosa\;  2023--2024")
CGG  = VGroup(copyright).rotate(PI/2).scale(0.5).to_edge(RIGHT).set_color(GRAY_D)
self.add(CGG)
\end{minted}


\begin{minted}[frame=lines,fontsize=\scriptsize,linenos]{python}
class ZCreditos(Scene):
    def construct(self):
        copyright = Tex(r"Copyright \textcopyright{\;} Marcos Bujosa\;  2023--2024")
        github = Tex(r"\texttt{https://github.com/mbujosab}").next_to(copyright, DOWN)
        CGG  = VGroup(copyright,github).scale(1.1)
        self.add(CGG)
        self.wait(6)
\end{minted}

\section{Rodando: 1,2,3\dots{} ¡acción!}
\label{sec:org6f856e9}

\begin{minted}[frame=lines,fontsize=\scriptsize,linenos]{sh}
echo $escena | manim -pql $fichero.py --disable_caching
\end{minted}


\begin{minted}[frame=lines,fontsize=\scriptsize,linenos]{sh}
echo $escena | manim -qh $fichero.py --disable_caching
\end{minted}


\begin{minted}[frame=lines,fontsize=\scriptsize,linenos]{sh}
rm -f $nombre.mp4 list.txt
for f in $subdir/$nombre/$calidad/*.mp4 ; do echo file \'$f\' >> list.txt; done && ffmpeg -f concat -safe 0 -i list.txt -c copy $nombre.mp4 && rm list.txt
\end{minted}

\begin{minted}[frame=lines,fontsize=\scriptsize,linenos]{sh}
rm -f $nombre.mp4 list.txt
for f in $subdir/$nombre/$calidad/*.mp4 ; do echo file \'$f\' >> list.txt; done && ffmpeg -f concat -safe 0 -i list.txt -c copy $nombre.mp4 && rm list.txt
\end{minted}
\end{document}